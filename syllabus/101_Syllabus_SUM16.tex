\documentclass[11pt]{article}
\usepackage[table,xcdraw]{xcolor}
\usepackage[colorlinks=true]{hyperref}
\hypersetup{colorlinks,urlcolor=blue}
\usepackage{booktabs}
\newcolumntype{L}[1]{>{\raggedright\arraybackslash}p{#1}}
\newcolumntype{C}[1]{>{\centering\arraybackslash}p{#1}}
\newcolumntype{R}[1]{>{\raggedleft\arraybackslash}p{#1}}

\newcommand{\hl}{\begin{flushleft}
	\rule{\textwidth}{1}
\end{flushleft}
}


\newcommand{\n}{\vspace{12pt}}
\usepackage{amssymb} 
\usepackage{verbatim}
\usepackage{amsmath}
\usepackage{graphicx}
\usepackage{geometry}



\parindent 0in


\begin{document}


\begin{center} UNIVERSITY OF NORTH CAROLINA\\
DEPARTMENT OF ECONOMICS \\
\end{center}

\begin{center}\textbf{{\hspace{-.05in}}
\underline{ECON 101-002: INTRODUCTION TO ECONOMICS}}
\\ Summer I 2016
\\ M - F 9:45AM - 11:15AM
\\ Gardner 307
\end{center}


\textbf{Instructor:} David A. D\'iaz \hspace{4.5cm} \textbf{Email:} \url{diazda@live.unc.edu}
\textbf{Office:} Phillips Annex 202 \\
\textbf{Office Hours:} Tuesday and Friday from 11:30am to 1:00pm, and by appointment.\\
\textbf{Course Website:} \url{http://sakai.unc.edu} \\
The Sakai web site contains the official course gradebook, announcements, and other supplementary materials.\\

\textbf{Course Objectives:} The purpose of this course is to introduce you to a new way of looking at the world. The course focuses on core economic concepts and introduces you to some basic models that economists use to make sense of what they observe around them. We will also explore how economists analyze the impact of different policies within the context of these models and determine whether the policy will have the intended - or an unintended - outcome. Throughout the course, I hope to increase your interest in economics and the role it can play in your everyday decision making.
\\

\textbf{Recommended Text:} N. Gregory Mankiw, Principles of Economics, $7^{th}$ Edition.\\

\textbf{Exam Dates:} \\
May 27  \hspace{3.07cm} Exam 1, 9:45AM - 11:15AM  \\
June 9  \hspace{3.2cm} Exam 2, 9:45AM - 11:15AM \\
June 15 \hspace{3cm} Final Exam, 8:00AM - 11:00AM \\
 
\textbf{Exam Format:} Each exam will cover material presented in class, readings scheduled outside of class, and homework assignments. The exam format will be provided a few days prior to a given exam. You should bring a pencil, scantron, and a \textbf{basic} calculator with you to each exam (an example of what constitutes a basic calculator is provided on Sakai -- if in doubt about yours, ask!). Students that arrive more than 30 minutes late will not be permitted to take the exam. 

\newpage
\textbf{Homework:} There will be 5 homework assignments, each due by 1:00PM on the assigned date. Homework must be turned into me directly - not slid under my door, placed in my mailbox, emailed, etc., unless prior approval has been granted. \textbf{Late homework will not be accepted}. The homework assignments are intended to be good study guides for the exams and will contain a mix of multiple choice and free response questions. I recommend you look at all assignments ahead of time and plan accordingly. You are free to work with your peers on the homework, but you each must turn in an individual assignment reflecting your own work. \\

\textbf{Attendance \& Participation:} Regular attendance and participation is strongly recommended. You are responsible for any announcements that you may have missed. You should attempt to get these notes/announcements from one of your peers before seeing me. It is expected that you will respect your peers and the instructor with appropriate behavior while in class and that you will arrive to class on time. You should refrain from browsing the web, texting, etc. during class time. Cellphones and laptops are not needed during class, so please put them away while in lecture.  \\

\textbf{Grading:}
\begin{itemize}
\item Homework: 20\%
\item Exam 1: 	22.5\%
\item Exam 2: 22.5\%
\item	Final Exam: 35\% 
\end{itemize}

The grading scale is as follows:
\begin{center}
\begin{tabular}{ p{3.5cm} p{3.5cm} }
A: 93 -- 100 &  C+: 77 -- 79.99\\
A--: 90 -- 92.99 & C: 70 -- 76.99\\
B+: 87 -- 89.99 & C--: 67 -- 69.99\\
B: 83 -- 86.99 & D+: 64 -- 66.99\\
B--: 80 -- 82.99 & D: 60 -- 63.99\\
 & F: $<$ 60
 

\end{tabular}
\end{center}

\newpage
\textbf{Missed Exams:} There are no make-up midterm examinations. If you must miss a midterm exam, you may be permitted to transfer the missed credit to the final examination. To qualify for a transfer of credit, you must contact me before the start of the missed midterm examination and provide me with an acceptable explanation. If the reason for your absence could not be foreseen, please make the request as soon as possible thereafter. All requests should be in writing and you may be asked to provide support with suitable documentation. If approved, the weight of the midterm will be placed on the final exam. \\

Regarding attendance for the final exam, the Chair of the Economics Department asks summer instructors not to change the date of a final exam unless the student has a compelling and documented personal reason.  In that case, the make-up exam will be administered on a date later than the scheduled final.  It will be set at a date and time that is convenient for both student and instructor.\\

\textbf{Academic Integrity:} All students are expected to adhere to the Honor Code  \\ (\url{http://instrument.unc.edu}). 
\begin{itemize}
	\item You may use your notes, textbooks, and classmates to complete homework assignments.
	\item No assistance is permitted on exams. During the exam, however, feel free to ask your instructor for clarification. The use of cell phones, computers, or any other unauthorized device (e.g., an unapproved calculator) during examinations is an explicit violation of the honor code.
\end{itemize}

\newpage



\begin{table}[h!]
	\centering
\begin{tabular}{@{}|l|l|l|l|@{}}
	\toprule
	\textbf{Date} & \textbf{Topic}                                                                           & \textbf{Readings}  & \textbf{Notes} \\ \midrule
	5/11          & Introduction to Economics                                                                & Ch. 1; Ch. 2              &     Skip section 2.1d         \\ \midrule
	5/12          & Trade                                                                                    & Ch. 3              &                \\ \midrule
	5/13          & Supply and Demand                                                                        & Ch. 4.1 -- 4.3          &            \\ \midrule
	5/16          & Market Equilibrium and Efficiency                                                & Ch. 4.4; Ch. 7              &  \\ \midrule
	5/17          & Elasticity                                                                               & Ch. 5              &       Homework 1 Due         \\ \midrule
	5/18          & Gov. Policy and Welfare                                                                  & Ch. 6, Ch. 8       &                \\ \midrule
	5/19          & Externalities; Public Goods                                                           & Ch. 10; Ch. 11     &                \\ \midrule
	5/20          & Costs of Production                                                                      & Ch. 13             &  \\ \midrule
	5/23          & Perfect Competition                                                                      & Ch. 14             &      Homework 2 Due          \\ \midrule
	5/24          & Monopoly                                                                                 & Ch. 15             &                \\ \midrule
	5/25          & \begin{tabular}[c]{@{}l@{}}Monopolistic Competition;\\ Oligopoly\end{tabular}            & Ch. 16; Ch. 17     &                \\ \midrule
	5/26          & Exam Review                                                                              &                    & Homework 3 Due \\ \midrule
	\rowcolor[HTML]{009901} 
	5/27          & \multicolumn{3}{c|}{\cellcolor[HTML]{009901}Exam 1}                                                                            \\ \midrule
	5/30          & Holiday                                                                                  &                    &                \\ \midrule
	5/31          & Introduction to Macroeconomics                                                           & Ch. 23             &                \\ \midrule
	6/1           & \begin{tabular}[c]{@{}l@{}}The Cost of Living; \\ Economic Growth\end{tabular}           & Ch. 24; Ch. 25     &                \\ \midrule
	6/2           & The Solow Model                                                                          & Solow Model videos &    \href{https://www.youtube.com/watch?v=gm2d5s3ZWYU}{Video 1}; \href{https://www.youtube.com/watch?v=e7xBy7zvyN8}{Video 2}  \\ \midrule
	6/3           & \begin{tabular}[c]{@{}l@{}}Savings, Investment, \\ and the Financial System\end{tabular} & Ch. 26             & Homework 4 Due \\ \midrule
	6/6           & \begin{tabular}[c]{@{}l@{}}Unemployment; \\ The Monetary System\end{tabular}                 & Ch. 28; Ch. 29     &                \\ \midrule
	6/7           & Money Growth and Inflation                                                               & Ch. 30             &                \\ \midrule
	6/8           & Exam Review                                                                              &                    & Homework 5 Due \\ \midrule
	\rowcolor[HTML]{009901} 
	6/9           & \multicolumn{3}{c|}{\cellcolor[HTML]{009901}Exam 2}                                                                            \\ \midrule
	6/10          & Aggregate Demand and Supply                                                              & Ch. 33             &                \\ \midrule
	6/13          & \begin{tabular}[c]{@{}l@{}}Monetary and Fiscal Policy; \\ Exam Review\end{tabular}                                                     & Ch. 34             &                \\ \midrule
	\rowcolor[HTML]{009901} 
	6/15          & \multicolumn{3}{c|}{\cellcolor[HTML]{009901}Final Exam}                                                                        \\ \bottomrule
\end{tabular}
\end{table}
\end{document}
