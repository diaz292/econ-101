\documentclass[11pt]{article}
\usepackage[table,xcdraw]{xcolor}
\usepackage[colorlinks=true]{hyperref}
\hypersetup{colorlinks,urlcolor=blue}
\usepackage{booktabs}
\newcolumntype{L}[1]{>{\raggedright\arraybackslash}p{#1}}
\newcolumntype{C}[1]{>{\centering\arraybackslash}p{#1}}
\newcolumntype{R}[1]{>{\raggedleft\arraybackslash}p{#1}}

\newcommand{\hl}{\begin{flushleft}
	\rule{\textwidth}{1}
\end{flushleft}
}


\newcommand{\n}{\vspace{12pt}}
\usepackage{amssymb} 
\usepackage{verbatim}
\usepackage{amsmath}
\usepackage{graphicx}
\usepackage{geometry}



\parindent 0in


\begin{document}


\begin{center} UNIVERSITY OF NORTH CAROLINA\\
DEPARTMENT OF ECONOMICS \\
\end{center}

\begin{center}\textbf{{\hspace{-.05in}}
\underline{ECON 101-002: INTRODUCTION TO ECONOMICS}}
\\ Summer I 2017
\\ M - F 9:45AM - 11:15AM
\\ Dey Hall 403
\end{center}


\textbf{Instructor:} David A. D\'iaz \hspace{4.5cm} \textbf{Email:} \url{diazda@live.unc.edu}
\textbf{Office:} Phillips Annex 103A \\
\textbf{Office Hours:} Tuesday and Friday from 11:30am to 1:00pm, and by appointment.\\
\textbf{Course Website:}
\href{https://sakai.unc.edu/portal/site/2c4d5bc8-5222-46ac-bcad-18681c7e9ce9}{https://sakai.unc.edu}\\
The Sakai web site contains the official course gradebook, announcements, and other supplementary materials.\\

\textbf{Course Objectives:} The purpose of this course is to introduce you to a new way of looking at the world. We will focus on core economic concepts and introduce you to some basic models that economists use to make sense of what they observe around them. We will also explore how economists analyze the impact of different policies within the context of these models and determine whether the policy will have the intended - or an unintended - outcome. Throughout the course, I hope to increase your interest in economics and the role it can play in your everyday decision making.
\\

\textbf{Recommended Text:} N. Gregory Mankiw, Principles of Economics, $8^{th}$ Edition.\\


\textbf{Course Weights:}
\begin{itemize}
	\item Participation: 10\%
	\item Homework: 25\%
	\item Exam 1: 	20\%
	\item Exam 2: 20\%
	\item	Final Exam: 25\% 
\end{itemize}

\textbf{Grading Scale:}

\begin{center}
	\begin{tabular}{ p{3.5cm} p{3.5cm} }
		A : 93 -- 100 &  C+ : 77 -- 79.99\\
		A-- : 90 -- 92.99 & C : 73 -- 76.99\\
		B+ : 87 -- 89.99 & C-- : 70 -- 72.99\\
		B : 83 -- 86.99 & D+ : 65 -- 69.99\\
		B-- : 80 -- 82.99 & D : 60 -- 64.99\\
		& F : $<$ 60
		
		
	\end{tabular}
\end{center}


\newpage

\textbf{Participation:} During class, we may work on various problems both as a class and individually with your classmates. Your participation grade will be based on your attendance in addition to your effort level when these group problems are posed. Browsing the web, talking about non-class related topics, playing on your phone, talking while the instructor is talking, etc. will negatively impact your grade.
\\

\textbf{Problem Sets:} In order for you to practice what we learn in class, there are various problem sets posted on the Sakai course site along with their solutions. You are encouraged to go through these problems both individually and in groups in order to ``know what you don't know.'' Note that these problem sets \textit{will not be collected or graded}, but rather are intended to be good study guides for the exams and contain a mix of multiple choice and free response questions similar to what you might expect to see on homework assignments or exams. \\

\textbf{Homework:} There are six graded homeworks each corresponding to 3-4 topics that will be due during the semester. These assignments are open book/note/classmate. Homeworks are due by 11:55PM on the dates noted in the course outline below. \textbf{No late submissions will be accepted}.  \\

\textbf{Exams:} Each exam will cover material presented in class, readings scheduled outside of class, and homework assignments/quizzes. The exam format will be provided a few days prior to a given exam. You should bring a pencil, scantron, and a \textbf{basic} calculator with you to each exam (an example of what constitutes a basic calculator is provided on Moodle -- if in doubt about yours, ask!). Students that arrive more than 30 minutes late will not be permitted to take the exam. 
\\

\textbf{Exam Dates:} \\
May 30  \hspace{5.31cm} Exam 1\\
June 9 \hspace{5.45cm} Exam 2\\
June 21  \hspace{5.25cm} Final Exam (8:00AM - 11:00AM) \\

\textbf{Exam Format:} Each exam will cover material presented in class, readings scheduled outside of class, and homework assignments. The exam format will be provided a few days prior to a given exam. You should bring a pencil, scantron, and a \textbf{basic} calculator with you to each exam (an example of what constitutes a basic calculator is provided on Sakai -- if in doubt about yours, ask!). Students that arrive more than 30 minutes late will not be permitted to take the exam. 



\newpage
\textbf{Missed Exams:} There are no make-up midterm examinations. If you must miss a midterm exam, you may be permitted to transfer the missed credit to the final examination. To qualify for a transfer of credit, you must contact me before the start of the missed midterm examination and provide me with an acceptable explanation. If the reason for your absence could not be foreseen, please make the request as soon as possible thereafter. All requests should be in writing and you may be asked to provide support with suitable documentation. If approved, the weight of the midterm will be placed on the final exam. \\

Regarding attendance for the final exam, the summer policy for missed final exams is the same as during the fall and spring semesters. In summary, you must have an excused final exam absence from a dean's office or student health in order to make-up the final exam at a mutually agreed upon time. See the \href{http://catalog.unc.edu/policies-procedures/attendance-grading-examination/#Final Examinations}{full policy} for more details: \\

\textbf{Academic Integrity:} All students are expected to adhere to the \href{http://instrument.unc.edu}{UNC Honor Code}. 
\begin{itemize}
	\item You may use your notes, textbooks, and classmates to complete homework assignments.
	\item No assistance is permitted on exams. During the exam, however, feel free to ask your instructor for clarification. The use of cell phones, computers, or any other unauthorized device (e.g., an unapproved calculator) during examinations is an explicit violation of the honor code.
\end{itemize}

\textbf{Need Help?} The EconAid Center, which is located in Gardner 009, will be staffed Monday through Friday during the first summer school session.  You are encouraged to stop by and ask the person on duty for assistance and/or use this space for a personal or group study space. The hours for the EconAid Center are:
\begin{itemize}
	\item Mondays (1pm-3pm)
	\item Tuesdays (1pm-5pm)     
	\item Wednesdays (1pm-5pm)
	\item Thursdays (1pm-5pm)
	\item Fridays (1pm-3pm)
\end{itemize} 

     


\newpage



\begin{table}[h!]
	\centering
\begin{tabular}{@{}|l|l|l|l|@{}}
	\toprule
	\textbf{Date} & \textbf{Topic}                                                                           & \textbf{Readings (Mankiw)}  & \textbf{Notes} \\ \midrule
	5/17          & Introduction to Economics                                                                & Ch. 1; Ch. 2              &        \\ \midrule
	5/18          & Trade                                                                                    & Ch. 3              &                \\ \midrule
	5/19          & Supply and Demand                                                                        & Ch. 4.1 -- 4.3          &            \\ \midrule
	5/22          & Market Equilibrium and Efficiency                                                & Ch. 4.4; Ch. 7              &  \\ \midrule
	5/23          & Elasticity                                                                               & Ch. 5              &       Homework 1 Due         \\ \midrule
	5/24          & Gov. Policy and Welfare                                                                  & Ch. 6, Ch. 8       &                \\ \midrule
	5/25          & Externalities; Public Goods                                                           & Ch. 10; Ch. 11     &                \\ \midrule
5/26          & Exam Review                                                                              &                    & Homework 2 Due \\ \midrule
5/29          & Holiday                                                                                  &                    &                \\ \midrule
\rowcolor[HTML]{009901} 
5/30         & \multicolumn{3}{c|}{\cellcolor[HTML]{009901}Exam 1}                                                                            \\ \midrule
5/31          & Costs of Production                                                                      & Ch. 13             &  \\ \midrule
	6/1          & Perfect Competition                                                                      & Ch. 14             &              \\ \midrule
	6/2        & Monopoly                                                                                 & Ch. 15             &                \\ \midrule
	6/5          & \begin{tabular}[c]{@{}l@{}}Monopolistic Competition;\\ Oligopoly\end{tabular}            & Ch. 16; Ch. 17     &         Homework 3 Due         \\ \midrule

	6/6          & Introduction to Macroeconomics                                                           & Ch. 23             &                \\ \midrule
	6/7           & \begin{tabular}[c]{@{}l@{}}The Cost of Living; \\ Economic Growth\end{tabular}           & Ch. 24; Ch. 25     &                \\ 
	\midrule
	6/8          & The Solow Model                                                                          & \href{https://www.youtube.com/watch?v=gm2d5s3ZWYU}{Video 1}; \href{https://www.youtube.com/watch?v=e7xBy7zvyN8}{Video 2} &  Homework 4 Due   \\ \midrule
	6/9           & Exam Review                                                                              &                    & \\ \midrule
\rowcolor[HTML]{009901} 
6/12           & \multicolumn{3}{c|}{\cellcolor[HTML]{009901}Exam 2}                                                                            \\ \midrule
	6/13           & \begin{tabular}[c]{@{}l@{}}Savings, Investment, \\ and the Financial System\end{tabular} & Ch. 26             &  \\ \midrule
	6/14           & \begin{tabular}[c]{@{}l@{}}Unemployment; \\ The Monetary System\end{tabular}                 & Ch. 28; Ch. 29     &                \\ \midrule
	6/15           & Money Growth and Inflation                                                               & Ch. 30             &       Homework 5 Due         \\ \midrule
	6/16          & Aggregate Demand and Supply                                                              & Ch. 33             &                \\ \midrule
	6/19          & \begin{tabular}[c]{@{}l@{}}Monetary and Fiscal Policy; \\ Exam Review\end{tabular}                                                     & Ch. 34             &            Homework 6 Due    \\ \midrule
	\rowcolor[HTML]{009901} 
	6/21         & \multicolumn{3}{c|}{\cellcolor[HTML]{009901}Final Exam, 8:00 - 11:00}                                                                        \\ \bottomrule
\end{tabular}
\end{table}
\end{document}
