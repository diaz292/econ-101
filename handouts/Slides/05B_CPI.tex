\documentclass[xcolor={dvipsnames},pdf, hyperref={colorlinks=true, citecolor=ForestGreen, linkcolor=BlueViolet, urlcolor=Magenta}]{beamer}
\usetheme{Frankfurt}  
\usecolortheme{whale}
\usepackage{tikz} 
\usepackage{graphicx}
\usepackage{dsfont}
\usepackage{hyperref}
\usepackage{alltt}
\usepackage{enumerate}
\usepackage{amsthm}
\theoremstyle{definition}
\newtheorem{exmp}{Example}[section]
\usepackage{verbatim}               % useful for \begin{comment} and \end{comment}
\usepackage{eurosym}                % used for euro symbol
\usepackage{caption} 
\usepackage{graphicx}
\usepackage{adjustbox}
\graphicspath{{Figures/}}
\usepackage{subcaption}
\usepackage{color}
\usepackage{float}
\usepackage{amssymb}
\usepackage{sgamevar}
\usepackage{remreset}% tiny package containing just the \@removefromreset command
\makeatletter
\@removefromreset{subsection}{section}
\makeatother
\setcounter{subsection}{1}

\newcommand{\defn}[1]{\textbf{#1}}


%Instructor version
\newcommand{\blank}[0]{}
\newcommand{\ddp}[1]{{\textcolor{ForestGreen}{#1}}} 
\newcommand{\dd}[1]{{\underline{\textcolor{ForestGreen}{#1}}}}

%Student version
%\newcommand{\blank}[0]{\vspace{2em}}
%\newcommand{\dd}[1]{\underline{\hspace{3cm}}} 
%\newcommand{\ddp}[1]{}

\addtobeamertemplate{navigation symbols}{}{%
	\usebeamerfont{footline}%
	\usebeamercolor[fg]{footline}%
	\hspace{1em}%
	\insertframenumber/\inserttotalframenumber
}

\section{The Consumer Price Index}

%% preamble
\title{The Cost of Living}
\author{David A. D\'iaz}
\institute{UNC Chapel Hill}
\date{}

\AtBeginSection[] %Section links on slides

\begin{document} 
	
	\begin{frame}
		
		\titlepage
		
	\end{frame}


\begin{frame}{CPI}
\begin{itemize}
	\item GDP measures the quantity of goods and services that an economy produces. \item This section looks at the overall \textit{cost of living}. How far does a dollar get you in terms of purchasing power?
	\item \defn{CPI:} A measure of the overall price level measured by the cost of goods and services bought by a typical consumer (e.g., housing, gasoline, food, etc.)
	
\end{itemize}
\end{frame}

\begin{frame}{Calculating the CPI}

\subsubsection*{Calculating the CPI}

\begin{enumerate}
	\item Fix the basket 
	\item Find the prices at each point in time
	\item Compute the basket's cost: $P_t = \sum_i p_{it} \times q_{i}$, where $i$ represents each good in the index. The quantity refers to the number of each good contained in the basket. Remember that it stays fixed!
	\item Choose a base year and compute the index
\end{enumerate}
\end{frame}

\begin{frame}{Calculating the CPI}
\begin{itemize}
	\item The CPI in year $t$ is computed as 
	\[CPI_t = P_t/P_{base yr} \times 100\]
	\item \defn{Inflation Rate:} The percentage change in the price index from the preceding period.
	\item Just like with the GDP deflator, we can use the CPI to find the inflation rate:
	\[\pi_{t+1} = (CPI_{t+1} - CPI_t)/CPI_t \times 100\%\]
\end{itemize}
\end{frame}

\begin{frame}{Calculating the CPI}

\tiny
\begin{exmp} Table \ref{cpi} shows the prices of textbooks and movie tickets in an economy. Suppose that a basket contains 20 movie tickets and 10 textbooks. 2014 is the base year.
	
	\begin{table}[ht]
		\centering
		\caption{Production in a Simple Economy}
		\label{cpi}
		\begin{tabular}{ c|c|c}       
			
			Year & Movie Tickets & Textbooks  \\
			\hline
			2013 & \$10.00 & \$120.00 \\
			2014 & \$11.50 & \$130.00 \\
			2015 & \$12.00 & \$135.00 \\
		\end{tabular}
	\end{table} 
	How much did the basket cost in 2015? What was the CPI in each year? What is the inflation rate in 2014?
\end{exmp}
\scriptsize
\ddp{\pause $P_{2013} = 10 \times 20 + 120 \times 10 = \$1,400$. \\ $P_{2014} = 11.50 \times 20 + 130 \times 10 = \$1,530$ \\
	$P_{2015} = 12 \times 20 + 135 \times 10 = \$1,590$. \\
\pause	$CPI_{2013} = 1400/1530 \times 100 = 91.5$. \\ $CPI_{2014} = 1530/1530 \times 100 = 100$. \\ $CPI_{2015} = 1590/1530 \times 100 = 103.9$. \\
\pause $\pi_{2014} = (100-91.5)/91.5\times 100\% = 9.3\%.$}
\end{frame}


\begin{frame}{Issues with the CPI}
\begin{enumerate}
	\item Substitution bias: Buyers move towards goods that become relatively less expensive. Basket remains constant, so it ignores substitution towards these goods and overstates increases in the cost of living.
	\item New goods: As new goods are introduced, consumers have more choices and each dollar is worth more. Fixed baskets ignore this increase in value.
	\item Quality change: Hard to measure changes in quality.
\end{enumerate}
\end{frame}

\section{The Effects of Inflation}

\begin{frame}{Correcting for Inflation}
\begin{itemize}
	\item Due to inflation, a dollar in one year is not the same as a dollar the next. In order to compare the purchasing power of money between two different years, we need to perform a unit conversion. 
	\item Since the CPI gives a measure of the price level each year, we can use a ratio of the CPI to convert from year $X$ dollars to year $Y$ dollars as such:
	\[\text{Amount in yr. $Y$ dollars = Amount in yr. $X$ dollar} \times (CPI_Y/CPI_X)\]
	\item \defn{Indexation:} The automatic correction by law or contract of a dollar amount for the effects of inflation.
\end{itemize}
\end{frame}


\begin{frame}{Correcting for Inflation}
\begin{exmp} The average price of gas in 1981 was \$1.42 a gallon. Meanwhile, the average price of gas in 2005 was \$2.50. If the CPI in 1981 was 88.5 and the CPI in 2005 was 196.4, was gas more expensive in 1981 or 2005 after correcting for inflation? 
\end{exmp}
\ddp{\pause 1981 gas price in 2005 dollars = 1.42 $\times$ (196.4/88.5) = \$3.15/gallon. \\ Gas was more expensive in 1981.}
\end{frame}

\begin{frame}{Correcting for Inflation}
\begin{exmp}
	The CPI in 1980 was 90, while in 2000 it was 200. If economics majors on average made \$24,000 in 1980 and \$55,000 in 2000, are they better off today or in 1980?
\end{exmp}

\ddp{\pause 1980 salary in 2000 dollars = 24,000 $\times$ (200/90) = \$53,333. Better off today.}
\end{frame}


\begin{frame}{Readings and Assignments}
\begin{itemize}
	\item Today: Mankiw Ch. 24
	\item Next time: Mankiw 25
	\item Problem Set 4, section 3
\end{itemize}
\end{frame}

\end{document}