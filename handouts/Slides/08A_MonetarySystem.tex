\documentclass[xcolor={dvipsnames},pdf, hyperref={colorlinks=true, citecolor=ForestGreen, linkcolor=BlueViolet, urlcolor=Magenta}]{beamer}
\usetheme{Frankfurt}  
\usecolortheme{whale}
\usepackage{tikz} 
\usepackage{graphicx}
\usepackage{dsfont}
\usepackage{hyperref}
\usepackage{alltt}
\usepackage{enumerate}
\usepackage{amsthm}
\theoremstyle{definition}
\newtheorem{exmp}{Example}[section]
\usepackage{verbatim}               % useful for \begin{comment} and \end{comment}
\usepackage{eurosym}                % used for euro symbol
\usepackage{caption} 
\usepackage{graphicx}
\usepackage{adjustbox}
\graphicspath{{Figures/}}
\usepackage{subcaption}
\usepackage{color}
\usepackage{float}
\usepackage{amssymb}
\usepackage{sgamevar}
\usepackage{remreset}% tiny package containing just the \@removefromreset command
\makeatletter
\@removefromreset{subsection}{section}
\makeatother
\setcounter{subsection}{1}


\newcommand{\defn}[1]{\textbf{#1}}


%Instructor version
\newcommand{\blank}[0]{}
\newcommand{\ddp}[1]{{\textcolor{ForestGreen}{#1}}} 
\newcommand{\dd}[1]{{\underline{\textcolor{ForestGreen}{#1}}}}

%Student version
%\newcommand{\blank}[0]{\vspace{2em}}
%\newcommand{\dd}[1]{\underline{\hspace{3cm}}} 
%\newcommand{\ddp}[1]{}

\addtobeamertemplate{navigation symbols}{}{%
	\usebeamerfont{footline}%
	\usebeamercolor[fg]{footline}%
	\hspace{1em}%
	\insertframenumber/\inserttotalframenumber
}

\section{What is Money?}

%% preamble
\title{The Monetary System}
\author{David A. D\'iaz }
\institute{UNC Chapel Hill}
\date{}

\AtBeginSection[] %Section links on slides

\begin{document} 
	
	\begin{frame}
		
		\titlepage
		
	\end{frame}

\begin{frame}{The Monetary System}
\begin{itemize}
	\item The existence of money makes trade easier, especially in complex economies. A barter system has trouble allocating scarce resources because trade requires \dd{the double coincidence of wants}.
	\item \defn{Money:} The set of assets in an economy that people regularly use to buy goods and services from other people.
\end{itemize}
\end{frame}


\begin{frame}{The Monetary System}
\begin{itemize}
	\item \defn{Medium of exchange:} An item buyers give to sellers when they want to purchase goods and services.
	
	\item \defn{Unit of account:} The yardstick used to post prices and record debts. Money measures and records economic value.
	
	\item \defn{Store of value:} An item people can use to transfer purchasing power from the present to the future.


\end{itemize}
\end{frame}

\begin{frame}{The Monetary System}
\begin{itemize}

	
	\item Because money is the economy's medium of exchange, it is the most \dd{liquid} asset.
	\item \defn{Liquidity:} The ease with which an asset can be converted into the economy's medium of exchange.
	
	
\end{itemize}
\end{frame}

\begin{frame}{The Monetary System}
\begin{itemize}
	\item The liquidity of an asset must be balanced against the asset's usefulness as a store of value. As we saw already, when prices rise, the value of money \dd{falls}. 
	
	\item \defn{Commodity money:} Money that takes the form of a commodity with intrinsic value (e.g., gold, cigarettes)
	
	\item \defn{Fiat money:} Money without intrinsic value that is used as such because of government decree. Acceptance of fiat money also depends on expectations and social conventions.
\end{itemize}
\end{frame}

\begin{frame}{The Monetary System}
\begin{itemize}
	\item \defn{Currency:} The paper bills and coins in the hands of the public.
	
	\item \defn{Demand deposits:} Balances in bank accounts that depositors can access on demand.
	
	\item Measures of the money stock:
	\begin{enumerate}
		\item M1 money supply: Includes currency, demand deposits, and traveler's checks
		\item M2 money supply: Includes M1, plus savings accounts, small-denomination times deposits, and money market deposit accounts.
		\item M3 money supply: Includes M2, plus large-denomination time deposits.
	\end{enumerate}
\end{itemize}
\end{frame}

\section{The Money Supply}

\begin{frame}{The Federal Reserve}
\begin{itemize}
	\item \defn{Central bank:} An institution designed to oversee the banking system and regulate the quantity of money in the economy.
	
	\item \defn{The Federal Reserve:} The central bank of the United States.
	
	\item Main roles of the Fed:
	\begin{enumerate}
		\item Regulation: Monitors banks' financial situation and facilitates transactions. Acts as lender of last resort.
		\item Monetary policy: Setting of the money supply (the quantity of money available in the economy).
	\end{enumerate}
\end{itemize}
\end{frame}

\begin{frame}{Banks and the Money Supply}
\begin{itemize}
	\item \defn{Reserves:} Deposits that banks have received but not loaned out.
	
	\item If a system is such that \textit{all} deposits are held as reserves, then it is a \dd{100\%} reserve banking system. 
	
	
	\item T-Account Representation:
	
	\begin{table}[ht]
		\centering
		\begin{tabular}{c|c }        
			
			Assets & Liabilities \\
			\hline
			\ddp{\$1,000 (reserves)}	&  \ddp{\$1,000 (deposits)}\\
			
			& \\
			
		\end{tabular}
	\end{table}
	
	\item Because each deposit in the bank reduces \dd{currency} and increases \dd{demand deposits} by the same amount, the money supply is \dd{unchanged} when individuals deposit money. 
	
\end{itemize}
\end{frame}

\begin{frame}{Banks and the Money Supply}
\begin{itemize}
	\item Instead of leaving all deposits idle, the bank can lend a portion of it out in order to make a profit by charging interest. 
	\item \defn{Fractional-Reserve banking:} A system in which banks hold only a fraction of deposits as reserves.
	
	\item \defn{Reserve ratio:} The fraction of deposits banks hold as reserves.
	
	\item T-Account Representation with a 15\% reserve ratio:
	
	\begin{table}[ht]
		\centering
		\begin{tabular}{ c|c }        
			
			Assets & Liabilities \\
			\hline
			\ddp{\$150 (reserves)}&  \ddp{\$1,000 (deposits)}\\
			
			\ddp{\$850 (loans)}	& \\
			
		\end{tabular}
	\end{table}
	
	\item Note what happens to the money supply: Depositors still have demand deposits of \$1,000, but now borrowers hold \dd{\$850} in currency. 
	
	
	\item When banks hold only a fraction of deposits in reserve, the banking system \dd{creates} money.
\end{itemize}
\end{frame}

\begin{frame}{Banks and the Money Supply}
\begin{itemize}
	\item Each time money is deposited and a bank loan is made, more money is created. 
	\item \defn{Money multiplier:} The amount of money that the banking system generates with each dollar of reserves. MM = 1/RR. As the RR increases, the MM decreases, and so the amount of MS creation decreases.
	
	\begin{exmp} 
		If the reserve ratio is 25\% and the central bank increases the quantity of reserves in the banking system by \$120, what is the maximum amount the money supply could increase?
	\end{exmp}

	\ddp{\pause $120 \times (1/.25) = 120 \times 4 = 480$.}
\end{itemize}
\end{frame}


\begin{frame}{Banks and the Money Supply}
\begin{itemize}
	\item The Fed has various means to influence the money supply, either directly or indirectly.

\end{itemize}
\end{frame}

\begin{frame}{Banks and the Money Supply}

	\begin{enumerate}
		\item Open-market operations: The purchase or sale of US government bonds by the Fed. \begin{itemize}
			\item Open-market purchases (buying bonds) increase the MS, 
			\item Open-market sales (selling bonds) decrease the MS. 
			\item This is the tool most often used
		\end{itemize}
	\end{enumerate}

\end{frame}

\begin{frame}{Banks and the Money Supply}

	\begin{enumerate}
		\setcounter{enumi}{1}
		\item Lending to banks: Banks borrow from the Fed's ``discount window.'' 
		\begin{itemize}
			\item Discount rate is the interest rate the Fed charges. 
			\item A lower discount rate encourages loans, which will increase reserves and thus the MS.
		\end{itemize}

	\end{enumerate}

\end{frame}

\begin{frame}{Banks and the Money Supply}

	\begin{enumerate}
		\setcounter{enumi}{2}
		\item Reserve requirements: Regulations on the minimum amount of reserves that banks must hold against deposits. 
		\begin{itemize}
			\item Increasing the RRR will decrease the number of loans, which will decrease the MS. This form is rarely used.
		\end{itemize}
	\end{enumerate}

\end{frame}

\begin{frame}{Banks and the Money Supply}
\begin{enumerate}
\setcounter{enumi}{3}
	\item Paying interest on reserves: By paying a higher interest rates on reserves, the Fed encourages banks to keep more in reserves and thus decreases MS.
	\item The federal funds rate: The interest rate at which banks make overnight loans to each other.
	\begin{itemize}
		\item  The Fed sets a target rate for this and influences it through open-market operations. 
		\item Buying bonds will increase reserves, which will decrease demand for overnight loans and thus decrease the federal funds rate.
	\end{itemize}
\end{enumerate}
\end{frame}

\begin{frame}{Banks and the Money Supply}
\begin{itemize}
	\item The Fed does not completely control the money supply. Issues:
\begin{enumerate}
	\item The Fed does not control the amount of money that \dd{households} actually deposit in banks. 
	\item The Fed does not control the amount of money that banks actually choose to lend. The reserve requirement is a \dd{minimum} amount that banks have to hold in deposits. Any deposits they hold over this amount are \dd{excess reserves}.
\end{enumerate}
\end{itemize}
\end{frame}

\begin{frame}{Readings and Assignments}
\begin{itemize}
	\item Today: Mankiw Ch. 29
	\item Next time: Mankiw Ch. 30
	\item Problem Set 6, section 1

\end{itemize}
\end{frame}

\end{document}